\documentclass[a4 paper, 12pt]{article}

\title{DECO2500 - INDIVIDUAL REPORT \\ Feedback 1}
\author{Tean-louise Cunningham (42637460)}
\date{17 April 2020}

\usepackage{geometry}
\geometry{margin=2cm}

\usepackage[utf8]{inputenc}
\usepackage[english]{babel}
\usepackage{xcolor}

\usepackage{appendix}
\usepackage{hyperref}
\hypersetup{
    colorlinks=true,
    linkcolor=black,
    filecolor=black,      
    urlcolor=blue,
}

%\usepackage{standalone}

\setlength{\parindent}{2em}
\setlength{\parskip}{1em}

\usepackage{enumitem}
\setlist{noitemsep, topsep=0pt}
\setlist[enumerate]{parsep=5pt} 

%%%%%%%%%%%%%%%%%%%%%%%%%%%%%%%%%%%%%%%%%%
\begin{document}

\section{Low Fidelity Prototype}
The initial research and conceptual design of the low-fidelity prototype were presented as a \href{run:./MindMap.pdf}{mind map} and \href{https://youtu.be/BRX7kF7ynSQ}{presentation}. In review, there are six main features that have been incorporated into the design of the low-fidelity prototype to address users needs. All of these features will be brought to the attention of the user during this first evaluation to determine they align with user needs and whether they should be carried into the next iteration.

\subsection{Choose Evaluation Method}
The purpose of these evaluations is to learn more about the users' needs, confirm that the conceptual model is appropriate for the users, and to provide feedback about design and flow. It is imperative that any misalignment of values or expectations are identified at this early stage before further time is spent on interaction design. Users must be to understand how the system works and it must align with their expectations to be a worthwhile project. The evaluation method chosen for the Low Fidelity Prototype is a combination of Design Walkthrough, Co-design and TAM.

A design walkthrough involves giving the user a task and, without guidance, ask them to complete the task. By observing and documenting how they interact with the system, feedback on how users expect the system to operate and what they they expect the system can be obtained. This feedback provides clearly whether the conceptual model chosen is appropriate to the users mental model. This method was chosen as the steps involved in using the application are almost the same for every instance, and so it is imperative that users are able to intuitively and easily complete these steps (i.e the task) at this early stage of design.

The co-design process generally involves explaining to the user how the system works and asking for their opinion how they would design the features of the application. For this evaluation, at points during the design walkthrough when a user gets stuck, in addition to asking them what the issues are and what they are experiencing, additional co-design practices will be adopted. This includes asking the user what they think should be happening and how they would design this part to be more intuitive. Since the user is in control of instructing the system it is important that they are able to achieve their goal of choosing a place to dine out the way they want to and expect, especially since it is a process that will be repeated on average twice a week for them. 

TAM consists of a set of questions based on perceived usefulness, perceived ease of use, attitude and intention to use the system. These questions are scaled from 1 (strongly disagree) to 4 (strongly agree). For this evaluation, eight of the questions were selected (at least one from each category). These questions were identified as most relatable to the purpose of the application, without being repetitive. The questions provide quantitative analysis that can assist with identifying problem areas of user acceptance, however by themselves they don't provide the reasoning behind the response. So in addition to these questions, follow up questions will be asked when a response less than strongly agree is selected to gain further insight into the users experience to understand why there is a gap between mental models. This method was incorporated as an extension to the design walkthrough/co-design process to determine that not only can users intuitively use the system but that they believe the design and the features assist them with mitigating the problem of deciding where to dine out.

Together these evaluation methods provide a succinct overview of whether at this stage of design that application gives the user what they want, what gaps may exist in the conceptual model and the overall acceptance of the design and flow of the prototype.


\subsection{Evaluation Protocol}
This protocol was created to provide structure and consistency amongst evaluation of participants. The protocol outlines the flow of the evaluation including scripts, instructions and details of notes to be taken. The protocol can be viewed as Appendix A.1

\subsection{Undertake Evaluations}
Due to current measures relating to COVID-19 all evaluations were performed online, unless part of the family unit. Users are invited to a Google Form where they are asked to sign in with their Google Account. From here they can navigate themselves through all aspects of the evaluation. The form can be viewed in Appendix A.2.  

Firstly, the user is introduced to the evaluation process and asked to complete a consent form online. The consent form is then uploaded in the provided section on the form. Secondly, the user is given instructions for the Design Walkthrough and directed, via a link, to a Google Slides presentation. Here they are given the task and access to navigate through slides depicting different pages of the paper prototype. The task is fairly vague to provide feedback on whether it is clear to the users what features are available without being told. The presentation is designed so that when users select areas of the paper prototype that are ‘clickable’ they are directed to the appropriate slide with the corresponding page. The presentation can be viewed in Appendix A.3.

Thirdly, whilst completing the task any time they are stuck for a period of time they are asked to stop and follow up questions are asked, including contribution of design as part of the co-design process. Finally, once the user has completed the task they select a link on the presentation that takes them back to the Google Form where they will complete the TAM evaluation. On the form, users will select their answer between 1 and 4 (strongly agree) which will be stored as quantitative results and follow up questions will be asked for further clarification. The results can be viewed in Appendix A.4.

Throughout all sections of this process, notes were taken of observations and feedback. These notes can be seen in Appendix A.5.

\subsection{Evaluation Analysis}
From the process of this evaluation, there are a number of key factors that will influence the design of the medium prototype to ensure increased usability and acceptance of the application for the user. The overview is separated by the key features of the application.

\begin{enumerate}
    \item Filter by preferences (including both craving and dietary requirements) - this filtering extends to the map results and menu display
        \begin{itemize}
            \item Users liked that they had the option to filter by dietary
            \item It wasn't clear that this extended to the menu page as well
        \end{itemize}

    \item Interactive map - replicate the familiar experience of exploring destinations
        \begin{itemize}
            \item Users had no problem selecting a restaurant
            \item All users selected the filter icon and expected to be taken back to the preferences page. At this time it was to bring up a more detailed version of the same list (with those previously chosen pre-selected). User suggestion was to have a more detailed list to begin with and the filter selection goes back to this page each time for familiarity.
        \end{itemize}

    \item Promote existing deals - have existing deals from restaurants separate from the menu and easily viewable based on date selection
        \begin{itemize}
            \item Users liked that deals was easily accessible and was one of the main tabs
            \item Was clear that it was filtered based on the day
        \end{itemize}

    \item Editable and shareable list - provide support to be able to compare options and share these with others
        \begin{itemize}
            \item Many of the users mentioned that before getting to the list page they didn't know what the 'list' icon on main tab meant. Once they reached the page they understood straight away what it meant. One of the suggestions was a different name, such as 'Compare' since list was vague. 
            \item Most of the users noticed the share button and knew exactly what it did, one of the users suggested adding the text 'share' as well like every other icon
            \item For all users, except one, when they reached the page it was not clear what the next step was. Their attention was on the 6 main tabs and they didn't notice the smaller heart and list icons in the top right corner. Many of the users didn't recognise what the 'list' icon was. One of the suggestion was to just have the text 'Add to list' and expected it to be near the name of the restaurant.
            \item Users expected to be able to select the restaurant and be taken back to the restaurant information page. At this time the selection on the list page was to select that restaurant to move forward. A suggestion made by a participant during the co-design was to have 'left swipe' to remove from list, 'click' to go to the restaurant page and 'right swipe' to be given directions.
        \end{itemize}

    \item Recommend to a friend - focus on word-of-mouth recommendations instead of star ratings
        \begin{itemize}
            \item Users liked the idea reviews were friends only
            \item Users understood the thumbs up and down was ratings, but not that it was friends only.
            \item None of the users had issues with the notification page, and understood it was related to friend recommendations (after being told earlier)
        \end{itemize}

    \item Restaurant information - ensure users are able to readily access general information about a restaurant without being overwhelmed 
        \begin{itemize} 
            \item Users understood all of the icons on the about page. 
            \item Users had no issues with finding the information about the restaurants.
        \end{itemize}
\end{enumerate}

\end{document}