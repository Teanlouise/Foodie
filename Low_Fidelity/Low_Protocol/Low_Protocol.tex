\documentclass[a4 paper, 10pt]{article}

\title{\textbf{EVALUATION PROTOCOL \\ Low Fidelity Prototype}\\}
\author{Tean-louise Cunningham (42637460)}
\date{}

\usepackage{geometry}
\geometry{margin=0.5cm}

\usepackage[utf8]{inputenc}
\usepackage[english]{babel}
\usepackage{xcolor}

\setlength{\parindent}{0em}
\setlength{\parskip}{1em}

\usepackage{enumitem}
\setlist{noitemsep, topsep=0pt}
\setlist[enumerate]{parsep=5pt} 


%%%%%%%%%%%%%%%%%%%%%%%%%%%%%%%%%%%%%%%%%%
\begin{document}
\maketitle
\begin{center}
Complete a design walkthrough with co-design and TAM questionnaire of a low-fidelity prototype to identify gaps between conceptual and mental models.
\end{center}

\section*{PREPARATION}
Since this is an individual evaluation only myself and the participant will be involved. Therefore, I will be fulfilling the role of facilitation, observation, recording and interaction flow. The following materials will be prepared for the user prior to the evaluation.
\begin{enumerate}
    \item Electronic Consent form
    \item Paper Prototype
    \item Walkthrough Presentation Slides
    \item Questionnaire
    \item Google Forms
    \item Zoom software
\end{enumerate}

\section*{INTRODUCTION}
    \subsection*{Opening Statement}
        \textcolor{lightgray}{User has been sent a link with survey and instructions on Google Forms. User’s screen is being shared over an online conference call.}

        \begin{itshape}
            Thank you for taking the time today to provide some feedback on the early stages of a mobile application. The purpose of this app is to assist you with deciding where to dine out using an interactive map, filtered preferences and comparison feature.

            Today, I will be showing you the basic prototype to observe how you interact with it , to determine any functionality or design that is not intuitive, and whether it is achieving its purpose effectively for you as the user.
        \end{itshape}

    \subsection*{Consent}
        \begin{itshape}
            Before we get started, please read carefully through this consent form. It reiterates the purpose for today and how your data will be used. Your personal details will not be used directly in any way and all observations are of your interaction with the software only. If you like to proceed with contributing please fill out this form and upload with the given link.
        \end{itshape}
        
        \textcolor{lightgray}
            {User reads through and fills out consent electronically with provided link and uploads.}
        
        \begin{itshape}
            Thanks for filling that out, please save it on your computer for the time being. If it any time you don’t wish to continue just let me know and we will stop, and none of your feedback will be used. A reminder that I am only testing the software and not evaluating you.
        \end{itshape}

\section*{DESIGN WALKTHROUGH}
    \subsection*{Instructions}
        \begin{itshape}
            To get your feedback, I will be asking you to complete a specific task using the prototype. At any point you get stuck or are confused I may pause you for a moment to ask you some questions. I won’t be explaining or showing you how to use the system. The point of this exercise is to see what you, as a first time user, expect of the system and how you think it should flow.

            In a moment you will be able to view the paper prototype and move through the pages. Please interact with the application as if it was reactive. This means pressing everything that you normally would to complete the task. The more realistic your interaction with the prototype the better the feedback to know where to improve. 

            You will have 10 minutes to complete the following task. Any questions? 

            Please click on the link to the presentation. The task is to choose two places and decide between them where you would like to eat dinner tonight, takeaway of course. You can start.
        \end{itshape}

        \textcolor{lightgray}{The user confirmed they have no questions and is starting the task.  Record, observe and take detailed notes of their process.}

    \subsection*{Task Notes}
        These are the steps that the user should be going through to complete the task, and observations relating to each one that need to be taken note of.
        \begin{enumerate}
            \item Filter preferences: 
            This is the default page and so all users will start here.
                \begin{itemize}
                    \item Do they know how to filter?
                    \item Did they fill all of the filters out before proceeding?
                    \item Did they know how to get to the next page?
                    \item How long did it take to complete this page?
                \end{itemize} 
            \item Interactive Map:
            This is the page that follows the preferences page.
                \begin{itemize}
                    \item Were they able to select a restaurant?
                    \item Did they know the map was interactive?
                    \item Did they try to press any other buttons on the page?
                    \item How long did it take them to select a restaurant?
                \end{itemize}
            \item Restaurant Information
                \begin{itemize}
                    \item After selecting a ‘dot’ on the interactive map they will be brought here. 
                    \item How many of the cards did they select?
                    \item Did they understand the menu was filtered?
                    \item Did they know what all the icons meant?
                    \item Were they able to add a place to a list?
                    \item What information did they want to look at?
                    \item How long did it take them to move to another step?
                \end{itemize}
            \item List page:
            If a user selects the ‘List’ icon they will be brought here to compare.
                \begin{itemize}            
                    \item Did they get to this page?
                    \item Do they know how to select a decision?
                    \item Do they understand what to do next?
                    \item How long did it take the user to find out their was a list page?
                \end{itemize}
            \item Repeat: 
            Since the task is to select 2 places, users will need to repeat 2-5
                \begin{itemize}
                    \item Were they able to find out how to get back to previous steps?
                    \item Did they want to choose a second place?
                    \item How long did it take to figure out how to get back to the map?
                \end{itemize}
            \item Recommendation Page:
            After they have chosen a place and completed the task they will be nudged here.
                \begin{itemize}
                    \item Did they understand what was happening?
                    \item Did they know what they were suppose to do?
                \end{itemize}        
        \end{enumerate}

\section*{CO-DESIGN}

    \subsection*{Instructions}
        \textcolor{lightgray}{While completing the task the user encounters a problem and is obviously stuck trying to move to the next step, or they took an action expecting different functionality. Prompt them to speak out loud during this time.}

        \begin{itshape}
        Please just pause for a moment:
            \begin{itemize}
                \item Do you understand what the next step is?
                \item What are you having trouble finding or understanding?
                \item Where/what do you think you should be able to find?
                \item How would you design this part?
            \end{itemize}
        \end{itshape}

        \textcolor{lightgray}{Show them the next step to continue the evaluation of the whole task.}

    \subsection*{Problem Notes}
    For each roadblock, in addition to noting the responses to the above questions:
    \begin{enumerate}
        \item The issue
            \begin{itemize}
                \item Do they understand what the next step is?
                \item What didn't they understand or couldn't find?        
                \item Did they get stuck because they didn’t understand the task?
                \item Did they get stuck because of the design?
                \item Was the flow confusing?
                \item After being showed the next step were they still confused?    
            \end{itemize}   
        \item Design Suggestions
            \begin{itemize}
                \item What do they think they should be able to find? 
                \item What were their suggestions to redesign?
                \item How was the experience prior to this point?
                \item What elements of the existing design did they like?
            \end{itemize}
    \end{enumerate}

\section*{TAM EVALUATION}
    \subsection*{Instructions}
        \textcolor{lightgray}{The user has completed the task.}

        \begin{itshape}
            Thank you for completing the task. Now select to go back to the form. Finally, I have some questions to rate your experience and your  acceptance of this application. The purpose is to determine the perceived usefulness and ease of use, your attitude towards the app and intention to use.

            For each question choose a number between 1 and 4, with 1 being strongly disagree and 4 being strongly agree. Please answer honestly. I may follow up with additional questions where necessary. 
        \end{itshape}

    \subsection*{Questionnaire}
        \begin{enumerate}
            \item I can accomplish deciding where to dine out more quickly using this application (PU1)
            \item This application enables me to make better decisions about where to dine out. (PU5)
            \item Overall I find this application useful (PU6)
            \item It is easy to use this application to decide where to dine out (PEOU2)
            \item Overall I believe this application is easy to use (PEOU3)
            \item Overall my attitude towards this application I favourable (ATT3)
            \item I will use this application on a regular basis in the future (ITO1)
            \item I will strongly recommend others to use this application (ITO3)
        \end{enumerate}

    \subsection*{Questionnaire notes}
        The quantitative answers from the users will be saved on Google Forms which automatically calculates and graphs collected data. Additionally, any score that is not 4 (strongly agree) will be followed up with the following questions.
        \begin{itemize}
            \item Why did you give this score?
            \item What stopped you from scoring higher?
        \end{itemize}


\section*{Conclusion}
    \begin{itshape}
        All done. Thank you so much for your time today. Just a reminder that if you would like to withdraw at any time, let me know and your data will not be used. Thank you for your time, it is greatly appreciated and your data is very valuable.
    \end{itshape}

\end{document}