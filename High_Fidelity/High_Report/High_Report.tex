% Setup
\documentclass[a4 paper, 12pt]{article}

% Title
\title{HIGH FIDELITY}

% Margins
\usepackage{geometry}
\geometry{margin=2cm}

% Images
\usepackage{graphicx}
\usepackage{float}
\usepackage[export]{adjustbox}
\setlength{\intextsep}{5pt plus 2pt minus 2pt}
\setlength\belowcaptionskip{0ex}
\usepackage[font=footnotesize,skip=2pt]{caption}

% Paragraph
\setlength{\parindent}{0em}
\setlength{\parskip}{1em}

% Text Formatting
\usepackage[utf8]{inputenc}
\usepackage[english]{babel}

% List spacing
\usepackage{enumitem}
\setlist{noitemsep, topsep=0pt}
\setlist[enumerate]{parsep=5pt} 

% Text Color
\usepackage{xcolor}

% Hyperlinks
\usepackage{hyperref}
\hypersetup{
    colorlinks=true,
    linkcolor=black,
    filecolor=black,      
    urlcolor=blue,
}

% Appendix
\usepackage{appendix}

% Include pdf
\usepackage{standalone}
\usepackage{pdfpages}

% Borders
\usepackage{mdframed}

% Symbols
\usepackage{amssymb}

\usepackage{multicol}

\usepackage{tabulary}
\usepackage{array}
\newcolumntype{L}{>{\arraybackslash}m{4cm}}
\newcolumntype{V}{>{\arraybackslash}m{6cm}}



\definecolor{mygreen}{HTML}{008037}
\definecolor{myblue}{HTML}{004AAD}
\definecolor{myorange}{HTML}{FF914D}
%%%%%%%%%%%%%%%%%%%%%%%%%%%%%%%%%%%%%%%%%%
\begin{document}
    
\section{High Fidelity Prototype}



    \subsection{Revised Requirements/Conception Design}

    \subsubsection*{System Concept Statement}
    
    The following updated metaphors will be applied to this next iteration.
        \begin{itemize}
            \item Delete: cross $\dashrightarrow$ rubbish bin
        \end{itemize}
    The following metaphors will be added to this iteration.
        \begin{itemize}
            \item Save: floppy disk
            \item Undo: Arrow arched to the left
            \item Delete all: Rubbish bin with three lines
        \end{itemize}

    \subsubsection*{Design Principles}
        \begin{itemize}
            \item minimal effort
            \item purposeful movement
            \item consistency
        \end{itemize}

    \subsubsection*{System Requirements}

    For the requirement 'Track user history - Remember preferences and customise experience' this will be now split into three parts. 
        \begin{itemize}
            \item Remember preferences - Shortcuts for expert users.
            \item Track user history - Easily re-visit restaurants.
            \item Favourite restaurants - Alternative way to search for options.
        \end{itemize}

    \subsection{Personas}
    \subsection{Interaction Scenarios}
    \subsection{UX Goals}
        \begin{itemize}
            \item I want to re-visit restaurants that I enjoyed.
            \item I want to dine out in my budget.
        \end{itemize}


    \subsection{Prototype Development}
    Using the medium fidelity prototype as a strong foundation and the updated conceptual design, further steps to improve the usability of the function were applied to a high fidelity prototype. These were the main issues from the medium prototype and how they will be resolved in the high fidelity prototype:
        % \begin{tabular}{|L|V|L|}
        %     \hline
        %     Issue & Modification & Result\\
        %     \hline \hline
        %     Now is confusing & Removed 'time' from main &  \\
        %     \hline            
        %     Overlay in list covers the name, unclear what selected & Replaced with overlay that keeps the name and details of selection and colours dark to show convention that it is actively selected. & \\
        %     \hline
        %     Colouring is no consistent & Pink outline is clickable button, Dark pink fill is active, light pink fill is selected & Navigation is clearer\\
        %     \hline
        %     Add to compare is hidden & Add unmovable action bar with this icon on restaurant page. & \\
        %     % \hline
        %     % Filter by budget hidden & Add price to main filter page & \textit{David now has clear option to filter by what is most important to him}\\
        %     % \hline
        %     % Difficulty setting default preferences & Replaced the icon with settings metaphor and added text, setting preferences is same as default filter page. & \textit{Sophie can now save her diet for next time}\\
        %     % \hline            
        %     % Overview on map & Classifications on interactive map & \textit{Matt can now choose in less time as desired.} \\
        %     % \hline
        %     % Go to restaurant without using compare feature & Add icon options to be able to get the information to go directly to a restaurant without adding to compare list. & \textit{Matt can now pick straight away like he mostly wants}\\
        %     % \hline
        %     Filter saved/history lists & Added filter bar to these page similar to the map and menu pages. & \textit{Jessica can now utilise her saved list better.}\\            
        %     \hline
        %     Use saved list to access restaurant information & Added option to go to restaurant page from saved list using overlay. &\textit{Jessica can now utilise her saved list better.}\\   
        %     \hline
        % \end{tabular}
      

    \begin{itemize}
        \item Filter by budget hidden $\dashrightarrow$ Add price to main filter page \\
        \textit{- David now has clear option to filter by what is most important to him}
        \item Difficulty setting default preferences $\dashrightarrow$ Replaced the icon with settings metaphor and added text, setting preferences is same as default filter page. \\
        \textit{- Sophie can now save her diet for next time}
        \item No overview on map  $\dashrightarrow$ Add classification icons to markers on interactive map            
        \item Can't go to restaurant without using compare feature $\dashrightarrow$ Add icon options to be able to get the information to go directly to a restaurant without adding to compare list. \\
        \textit{- Matt can now choose in less time as desired with just one option.}
        \item Expected to be able to filter saved/history lists $\dashrightarrow$ Added filter bar to these page similar to the map and menu pages. 
        \item Can't use saved list to access restaurant information $\dashrightarrow$ Added option to go to restaurant page from saved list using overlay.  \\
        \textit{- Jessica can now utilise her saved list better.}
    \end{itemize}



    Additional functionality has also been added to give the experts a more realistic experience of how the interface should behave.
    \begin{itemize}
        \item At least one option for each dropdown on main filter page.
        \item Basic preview of pages when adding default preferences.
        \item Change of state colour when adding a restaurant to compare or favourites.
        \item Preview of application status messages when saving a selection.
    \end{itemize}

    This is an overview of the updated prototype, with the important aspect changes outlined and explained visually. 
    \begin{figure} [H]
        \centering
        \includegraphics[width=\textwidth, frame]
            %{./High_Fidelity/High_Report/images/high_proto_notes.PNG}  
            {./images/high_proto_notes.PNG}
        \caption{High Prototype}
    \end{figure}  


    \subsection{Evaluation Methods}
    To effectively evaluate the high fidelity prototype, a heuristic evaluation will be undertaken. This method requires UI/UX experts to critically assess the interface of the application against a set of criteria to determine whether it meets a minimum standard of usability. This criteria will be a list of 10 heuristics which have been specifically selected for this application. By using experts there are fewer ethical and practical issues, and their knowledge can provide key insights into the general expectations of usability the domain and identify potential issues when all functionality is properly implemented. However, it is important to keep in mind that there is an increased possibility of trivial issues being identified and some larger issues overlooked as they are not evaluating through the eyes of the user.    

    There are two preparation steps before starting the heuristic evaluation. The first is to determine the features of the application. These have been outline and updated continuously in the system requirements sections of the reports. The second step is the choose the set of heuristics with these features in mind. By looking at the \textcolor{mygreen}{SMART} and \textcolor{myblue}{Nielson 2001} and \textcolor{myorange}{HOMERUN} heuristics, 10 heuristics were chosen as criteria to determine their usability.
        \begin{enumerate}
            \item \textcolor{mygreen}{Provide immediate notification of application status}: 
            This is a refinement of \textit{\textcolor{myblue}{Visibility of system status}}, which states 'The system should always keep users informed about what is going on, through appropriate feedback within reasonable time.' However, as this is a mobile application, the status must be immediate and due to screen size should be done non-intrusively where appropriate.
            \item \textcolor{mygreen}{Use a theme and consistent terms, as well as conventions and standards familiar to user}: This is a combination of \textit{\textcolor{myblue}{Consistency and standards (use platform conventions)}} and \textit{\textcolor{myblue}{Match between system and the real world (for user)}}. Essentially, users should know exactly what words and actions mean, and these phrases should be familiar to the user with information appearing in a logical order. Additionally, as this is a mobile application a theme should be used 'to ensure different screens look alike' and that the 'standards that users have come to expect in a mobile application' are used.         
            \item \textcolor{mygreen}{Prevent problems where possible; assist users should an error occur}: This is a combination of \textit{\textcolor{myblue}{Help users recognize, diagnose and recover from errors}} and \textit{\textcolor{myblue}{Error prevention}}. Error message should be clear and concise with solution suggestions, and even better 'prevents a problem from occurring in the first place'. It is essential that a mobile application 'is error-proofed as much as possible'.        
            \item \textcolor{myblue}{User control and freedom:} 'Users often choose system functions by mistake and will need a clearly marked "emergency exit" to leave the unwanted state without having to go through an extended dialogue. Support undo and redo.'        
            \item \textcolor{mygreen}{Each interface should focus on one task:} Due to the  of mobile application, This heuristic is unique to mobile application as a result of their use cases (frequent interruptions) and screen space (less cluttering). This means 'only having the absolute necessary elements onscreen to complete that task'.            
            \item \textcolor{myblue}{Aesthetic and minimalist design:} The original is \textit{\textcolor{myorange}{High quality content}} which is mainly for websites but highlights the important of providing functionality users want. The mobile equivalent is \textit{\textcolor{mygreen}{A visually pleasing interface}} and focus on the forgiveness of users if the interface is attractive. The 'attractiveness' of an application is a qualitative measure with different opinions. Instead this heuristic focuses on the aesthetic of the app by assessing whether it is minimalist, which is the preferred design of today's mobile Users (Greenlaugh, 2018).       
            \item \textcolor{myblue}{Recognition rather than recall:} The HOMERUN equivalent is \textit{\textcolor{myorange}{Ease of use}} which states that 'users need to be able to find the information they need quickly and easily. This is similar to \textit{\textcolor{mygreen}{Intuitive interfaces make for easier learning}}, which says similar for mobile interfaces in that they 'should be easy-to-learn whereby next steps are obvious'. Neither of these heuristics are clear about how the application should be achieving intuitiveness. Instead this heuristic focuses on minimises cognitive load by 'making objects, actions, and options visible' so that users dont need to remember each part of the process.
            \item \textcolor{mygreen}{Design a clear navigable path to task completion:} A more refined heuristic compared to \textit{\textcolor{myorange}{Relevant to users’ needs}} which measures whether the users are able to perform the task they want. This heuristic measures whether users are 'able to see right away how they can interact with the application and navigate their way to task completion'.     
            \item \textcolor{mygreen}{Allow configuration options and shortcuts:} A reworded revision of \textcolor{myblue}{Flexibility and efficiency of use}, it more appropriately identifies that the system should provide expert users with ability to tailor frequent actions. 
            \item \textcolor{mygreen}{Facilitate easier input:} This heuristic is unique to mobile applications in that it focuses on making it easy to input content from the perspective of a mobile device. 
        \end{enumerate}

    These were not chosen for this application:
        \begin{itemize}
            \item \textcolor{mygreen}{Display an overlay pointing out main features when appropriate or requested to help first-time users.:} This is similar to \textit{\textcolor{myblue}{Help and documentation}}, however there are no difficult elements of the application that need explaining and so no documentation is included for its use to be assessed by an expert
            \item \textcolor{mygreen}{Use camera, microphone and sensors to lessen user’s  workload:} The only sensors used in this application is GPS which is adopted from other applications and so there is no unique factors to asses for this application. According to research and existing solution no other sensors would be appropriate. 
            \item \textcolor{mygreen}{Cater for diverse mobile environments (lighting, ambient noise, gloves, etc):} At this stage there have been no accommodations made for different use case environments and so there is nothing for users to assess against this heuristic. 
            \item \textcolor{myorange}{Often updated, Minimal download time, Unique to the online medium, Net-centric corporate culture supporting site:} None of these heuristics are relevant to the mobile application or available to be assessed at this stage of prototyping.            
        \end{itemize}

    The are three stages of a heuristic evaluation; briefing session, evaluation period and debriefing session. During the evaluation period, experts go through the features of the application individually twice. The first time is to get a feel and understanding of the interface, and the second time through is to focus on the specific features and make notes. During this second pass, according to chapter 13.4.2 of The UX Book, each expert 'individually browses through each part of the interaction design, asking the heuristic questions about that part'. The expert takes note of where and how the heuristic has been violated, how this would cause usability issues for the user and the probable effect on the user. They also rate the severity of the usability issue choosing between two options for three factors; occurrence (common, rare), impact (low, high) and perseverance (very, not). The combination of these factors provide a severity rating (0 to 4), and the mean of at least three evaluators ratings is satisfactory to determine the seriousness of the usability issue.     
        \begin{figure} [H]
            \centering
            \includegraphics[width=0.6\textwidth, frame]
                %{./High_Fidelity/High_Report/images/high_proto_notes.PNG}  
                {./images/high_heuristic_severity.PNG}
            \caption{Heuristic Severity Ratings}
        \end{figure} 

    
    \subsection{Evaluation Protocol}
    The purpose of the protocol is the same as previous prototype evaluations. The protocol can be viewed in \hyperref[sec:B.1]{Appendix C.1}. To guide the experts through the three stages of the heuristic evaluation, after commencing the online call, the experts are provided with a link to a Google Forms. The form can be viewed in \hyperref[sec:C.2]{Appendix C.2}. This form  The first step is the briefing session where the expert is given an overview of the task and expectations, and asked to complete a consent form. The second stage is completing the heuristic evaluation. The expert is given a list of 13 tasks, each of which pass through every page and feature of the application. The expert is asked to complete each task at their own speed with guidance when/if and error occurs.  They are provided with a link to the high fidelity prototype on figma. The slides can be viewed in \hyperref[sec:C.3]{Appendix C.3}.
        
    After the user has completed all of the tasks and expresses they feel confident with the system they are directed back to the Google Forms to prepare for the second pass. During the second pass of the application, the expert is again asked to complete each of the tasks however this time they are to specifically evaluate the usability of the system against the chosen heuristics. There are ten heuristics that were identified as part of the preparation for the evaluation. On the Google Forms, the experts are provided a link to a Google Sheets where there is a tab called HEURISTICS which lists and describes each of the heuristics. In the sheet is a second tab for the expert to fill out their notes with the appropriate headings, including the severity rating factors. The tabs of the sheet can be viewed \hyperref[sec:C.4]{Appendix C.4}. The expert uses the same link to the prototype to evaluate each task against these heuristics. After they are satisfied they have identified all the current issues they are debriefed.
    
    This evaluation included 5 experts as this would identify at least 75\% of the evaluations (Lecture 10). One of the evaluators also took part in the low fidelity evaluation and another has participated in all three evaluations. This range of familiarity with the application may provide the identification of some unique issues and ensures that all previous issue were addressed. The raw notes are in \hyperref[sec:C.5]{Appendix C.5}. 

    \subsection{Evaluation Results}
    From the heuristic evaluation there were a range of issues identified. Where more than one evaluator has noted the issue the mean of the severity factor responses has been taken. Also in this case, the heuristic that was consistent amongst all answers was chosen or a judgment call was made for the most appropriate. \\
    \begin{tabular}{|V|l|l|l|l|}
        \hline
        \textbf{Issue} & \textbf{\#Experts} & \textbf{Heuristic} & \textbf{Factors} & \textbf{Severity} \\
        \hline \hline 
        \multicolumn{5}{|c|}{DROPDOWN} \\
            \hline \hline
            want to select whole box not just arrow & 4         & 10        & common,low,very   & major \\    
            \hline
            want to select text not just box        & 3         & 10        & common,low,very   & major \\
            \hline
            want to select anywhere to save option  & 3         & 3, 8      & common,low,very  & major \\
            \hline \hline        
        \multicolumn{5}{|c|}{MAIN} \\
            \hline \hline
            reset option                        & 2 & 3 & rare,low,very & cosmetic \\
            \hline
            home icon shouldn't be on this page & 1 & 3 & rare,low,very & Cosmetic \\
            \hline \hline
        \multicolumn{5}{|c|}{COMPARE} \\
            \hline \hline
            too many steps to delete, edit page redundant & 4 & 8,3 & common,low,very & major \\
            \hline
            save doesn't take user away from edit screen & 1 & 8 & common,high,very & catastrophic \\
            \hline \hline
        \multicolumn{5}{|c|}{RESTAURANT} \\
            \hline \hline
            no shortcut to compare/favourite & 2 & 9 & rare,low,not & cosmetic \\
            \hline 
            want to see more detail about reviews & 2 & 2 & common,low,not & minor \\
            \hline
            menu text area is small, lots scrolling & 1 & 7 & common,low,not & minor\\
            \hline \hline
        \multicolumn{5}{|c|}{RECOMMEND} \\
            \hline \hline
            saved message has to be clicked out of and then back & 2 & 8, 4 & common,low,very & major \\    
            \hline \hline
        \multicolumn{5}{|c|}{PROFILE} \\
            \hline \hline    
            not clear need to save defaults & 2 & 1,6 & common,high,very & catastrophic \\
            \hline
            cant add to favourite from history & 2 & 9 & rare,low,very & cosmetic \\
            \hline
            cant undo delete for saved & 1 & 4 & common,low,not &  minor \\
            \hline
            not able to delete/modify history & 1 & 4 & rare,low,not & cosmetic \\
            \hline    
    \end{tabular}    


    %\begin{itemize}
        % \item DROPDOWN (Main, Default, Deals)        
        %     \begin{itemize}
        %         \item \textcolor{myblue}{want to select whole box not just arrow} \textit{(4 experts - heuristic 10 - common,low,very)}
        %         \item \textcolor{myblue}{want to select text not just box} \textit{(3 experts - heuristic 10 - common,low,very)}                
        %         \item \textcolor{myblue}{want to select anywhere to save option} \textit{(3 experts - 4, 10, 3,8 - common,low,very)}
        %     \end{itemize}    
        % \item MAIN
        %     \begin{itemize}
        %         \item reset option - [2 experts - 3 - rare,low,very] - Cosmetic
        %         \item home icon shouldn't be on this page - [1 expert - 3 - rare,low,very] - Cosmetic
        %     \end{itemize}
        % \item COMPARE
        %     \begin{itemize}
        %         \item too many steps to delete, edit page redundant [4 experts  - 4,9, 7,8, 2, 3 - rare, low, very]  - Cosmetic
        %         \item save doesn't take user away from edit screen [1 expert - 8,10 - common,high,very] - Catastrophic
        %     \end{itemize}
        % \item RESTAURANT
        %     \begin{itemize}
        %         \item no shortcut to compare/favourite [2 experts - 9 - (rare,low,not)] - Cosmetic
        %         \item want to see more detail about reviews - [2 experts - 2 - common,low,not] - Minor 
        %     \end{itemize} 
        % \item RECOMMEND
        %     \begin{itemize}
        %         \item saved message has to be clicked out of and then back [2 experts - 8, 1,4 - common,low,very] - Major
        %     \end{itemize}
        % \item HISTORY
        %     \begin{itemize}
        %         \item not able to delete/modify - 4 (rare,low,not) - Cosmetic
        %         \item cant add to favourite - [2 experts - 9 - rare,low,very]
        %     \end{itemize}
        %     \item SAVED
        %         \begin{itemize}
        %             \item cant undo delete - 4 (common,low,not)
        %         \end{itemize}
        %     \item DEFAULT
        %         \begin{itemize}
        %             \item not clear need to save - 6 (common,high,very), 1 (rare,high,not)
        %         \end{itemize}
        %     \item MENU
        %         \begin{itemize}
        %             \item text area is small, lots scrolling - 7 (common,low,very)
        %         \end{itemize}
        % \end{itemize}


    % \begin{itemize}
        % \item DROPDOWN (Main, Default, Deals)
        %     \begin{itemize}
        %         \item want to select text not just box- 3,10 (common,low,very), 10 (common,low,very), 10 (common,low,very)
        %         \item want to select whole box not just arrow - 8,10 (common,high,very), 10 (common,low,very), 10 (common,low,very), 10 (rare/low/very)
        %         \item want to select anywhere to save option - 4 (rare,low,very), 10 (common,low,very), 3,8 (common,high,not)
        %     \end{itemize}
    %     \item MAIN
    %         \begin{itemize}
    %             \item reset option - 3 (rare,low,very), 3,4 (rare,low,very)
    %             \item home shouldn't be here - 3 (rare,low,very)
    %         \end{itemize}
    %     \item COMPARE
    %         \begin{itemize}
    %             \item too many steps to delete, edit page redundant - 4,9 (rare,low,very), 7,8	(rare,high,very), 2 (common,low,very), 3 (rare,high,very)
    %             \item save doesn't take user away from edit screen - 8,10 (common,high,very)
    %         \end{itemize}
    %     \item RESTAURANT
    %         \begin{itemize}
    %             \item no shortcut to compare/favourite - 9 (rare,high,very), 9 (rare,low,very)
    %             \item want to see more detail about reviews - 2,10 (common,high,not), 2 (common,high,not)
    %         \end{itemize} 
    %     \item HISTORY
    %         \begin{itemize}
    %             \item not able to delete/modify - 4 (rare,low,easy)
    %             \item cant add to favourite - 9 (rare,low,very), 9 (rare,high,not)
    %         \end{itemize}
    %     \item RECOMMEND
    %         \begin{itemize}
    %             \item saved message has to be clicked out of and then back - 8 (common,low,very), 1,4 (rare,low,easy)
    %         \end{itemize}
    %     \item SAVED
    %         \begin{itemize}
    %             \item cant undo delete - 4 (common,low,not)
    %         \end{itemize}
    %     \item DEFAULT
    %         \begin{itemize}
    %             \item not clear need to save - 6 (common,high,very), 1 (rare,high,not)
    %         \end{itemize}
    %     \item MENU
    %         \begin{itemize}
    %             \item text area is small, lots scrolling - 7 (common,low,very)
    %         \end{itemize}
    % \end{itemize}

	
 		


    \subsection{Evaluation Analysis}

     All of the heuristic violation results identified according to this sample of experts, so while approximately 75\% may have been collected there is potential that another set of the same evaluations with different experts could yield different results. Additionally, many of the issues identified were only identified by 1 or 2 experts which provides unreliable results. A further step that could be taken would be to recontact the experts with the entire list of issues and ask them to rate each one. This was not available at this time. 


\begin{itemize}
    \item (5) Each interface should focus on one task: This was the only heuristic that was not violated.
    \item (1) Provide immediate notification of application status \& (6) Recognition rather than recall: Both violated once by the same issue which occurred on both the default preferences and edit compare list page. Experts noted that it was not clear that on these edit pages the changes needed to be saved as not only was there no response from the application when the pages weren't saved but this was not required for similar behaviour in other areas of the application. 
    \item (2) Use conventions and standards familiar to the user: Experts noted the inability to view more information about reviews on the restaurant page by selecting the icons. This is an expected behaviour and functionality from other rating systems. This is a minor issue so it is low priority. $\dashrightarrow$ Support the display of more information when selecting this area i.e. which friends voted. The third issue 
    \item (4) User control and freedom: This heuristic was violated on the profile page by 2 seperate issues. The first is that once an option is deleted from the saved it cannot be undone if this was a mistake and the second is that the user is unable to delete/modify an option from their history if it is incorrect. $\dashrightarrow$ prompt user to confirm they want to deleted
    \item (7) Aesthetic and minimalist design: This was violated on the menu page according to one expert who stated the text was too small which would be problematic if the menu was long due to extended scrolling. 
    \item (9) Allow configuration options and shortcuts: Firstly, experts noted that there was no way to get to the compare or favourites page after adding a restaurant without going back to the map. Secondly, a restaurant in the history list could not be added to favourites without searching for it again $\dashrightarrow$ As it is cosmetic this would be one of the last things changed, but a shortcut such as a long press could be introduced for expert users. $\dashrightarrow$ add favourite action to overlay on history page
    \item (10) Facilitate easier input: The issues all relate to the selectability of the dropdown boxes and their content. Currently only selecting the arrow icon and the tick box exactly will open/select an option $\dashrightarrow$  These issues were all identified as having a major market impact i.e. 'important to fix, should be given high priority' (nngroup).  $\dashrightarrow$
    The whole box needs to be made selectable as well as being able to select the text of an option. 
    \item (3) Prevent problems where possible; assist users should an error occur: The first relates to the absence of a reset option on the main page. The second issue was that the home icon is evident on the page despite this being the home page, potentially confusing users with what the icon then means. Also, users options are not saved unless they select arrow exactly. $\dashrightarrow$ As they are cosmetic these issues would be fixed if there was time. $\dashrightarrow$ this would be resolved by added this option as an icon on this page that is less obvious than the GO button (as used less) and would prompt the user to confirm they want to reset.  $\dashrightarrow$This would be resolved by simply removing it.  $\dashrightarrow$
    Also, once an option has been selected the users should not to worry about how they exit the overlay and should be saved no matter their next click. All of these updates align with Fitts law as larger buttons/selection area and proximity between selections is better for the user. 
    \item (8) Design a clear navigable path to task completion: This issue relates to the process of deleting an option from the compare list. Currently users must select 'edit', then select the option, then select 'delete' then save then back. $\dashrightarrow$ rated a major usability problem reliably. This issue will definitely need to be rectified in the next iteration by removing the edit functionality. Instead a delete option, using the bin metaphor, will be added to the existing list of actions in the overlay when selecting an option on this page. This reduces the number of steps from 4 to 2 and replicates the same behaviour as the saved page. 
    \end{itemize}




    % From the results, the severity ratings of the three issues relating to the dropdown menus throughout the application can be considered reliable (more than 3 experts) are the issues relating to the . These issues were all identified as having a major market impact i.e. 'important to fix, should be given high priority' (nngroup). The issues all relate to the selectability of the dropdown boxes and their content. The whole box needs to be made selectable (not just the arrow icon) as well as being able to select not only the tick box but also the text of an option. By resolving these issues there will be no further violation of \textbf{heuristic 10 (facilitate easier input)} at this time. Also, once an option has been selected the users should not to worry about how they exit the overlay and should be saved no matter their next click. All of these updates align with Fitts law as larger buttons/selection area and proximity between selections is better for the user. By resolving these issues all of the vilations of heuristic 10 are resolved. 

    % The two issues relating to the main page were only identified by at most two experts, and they are both of cosmetic severity relating to \textbf{heuristic 3 (prevent problems where possible)}. As they are cosmetic these issues would be fixed if there was time. The first relates to the absence of a reset option, this would be resolved by added this option as an icon on this page that is less obvious than the GO button (as used less) and would prompt the user to confirm they want to reset. The second issue was that the home icon is evident on the page despite this being the home page, potentially confusing users with what the icon then means. This would be resolved by simply removing it.

    % There are two issues identified by experts on the COMPARE page, one of which almost all of the experts noted and as been rated a major usability problem reliably. This issue relates to the process of deleting an option from the compare list. Currently users must select 'edit', then select the option, then select 'delete' then save. All experts agreed this violates \textbf{heuristic 8 (Design a clear navigable path to task completion)} as well some stating it also violates \textbf{heuristic 3 (prevent problems where possible)}. This issue will definitely need to be rectified in the next iteration by removing the edit functionality. Instead a delete option, using the bin metaphor, will be added to the existing list of actions in the overlay when selecting an option on this page. This reduces the number of steps from 4 to 2 and replicates the same behaviour as the saved page. Also, users will be again be prompted to ensure they wish to delete this option to ensure this is their desired action, which is still clearer than having to save. The second issue of having to select the back arrow after saving the list, instead of save taking the user back, is an example of where 1 users response can be unreliable. The expert rated this issue catastrophic, despite the whole mess of the process only being considered major. Regardless, this issue will be resolved when the edit functionality is removed and simplified to the action button on the overlay. With these changes all of the violations of heuristic 3 will be resolved. 


    
    % On the restaurant page there are three issues, none of which have reliable severity ratings. The first is a cosmetic error, according to 2 experts, violating heuristic 9 (allowing configuration options and shortcuts). Experts noted that there was no way to get to the compare or favourites page after adding a restaurant without going back to the map. As it is cosmetic this would be one of the last things changed, but a shortcut such as a long press could be introduced for expert users. The second issue violates heuristic 2 (use conventions and standards familiar to user) as 
    
    
    













\end{document}


