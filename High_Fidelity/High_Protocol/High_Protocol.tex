\documentclass[a4 paper, 10pt]{article}

\title{\textbf{EVALUATION PROTOCOL \\ High Fidelity Prototype}}
\author{Tean-louise Cunningham (42637460)}
\date{}

\usepackage{geometry}
\geometry{margin=0.5cm}

\usepackage[utf8]{inputenc}
\usepackage[english]{babel}
\usepackage{xcolor}

\setlength{\parindent}{0em}
\setlength{\parskip}{1em}

\usepackage{enumitem}
\setlist{noitemsep, topsep=0pt}
\setlist[enumerate]{parsep=0pt} 

\usepackage{multicol}
\usepackage{amssymb}

%%%%%%%%%%%%%%%%%%%%%%%%%%%%%%%%%%%%%%%%%%
\begin{document}
\maketitle
\begin{center}
HCI Experts individually perform a Heuristic evaluation of the high fidelity prototype to determine as a non-use if it meets a standard of usability.
\end{center}

\section*{PREPARATION}
Since this is an individual evaluation only myself and the participant will be involved. Therefore, I will be fulfilling the role of facilitation, observation, recording and interaction flow. The following materials will be prepared for the user prior to the evaluation.

    \begin{multicols}{2}
        \begin{enumerate}
            \item Electronic consent form
            \item Digital high fidelity prototype
            \item figma prototype presentation
            \item Google forms with instructions
            \item Google sheets for heuristic
            \item Zoom software
        \end{enumerate}
    \end{multicols}

\section*{INTRODUCTION}

    \subsection*{Opening Statement}

        \textcolor{lightgray}{User has been sent a link with survey and instructions on Google Forms. User’s screen is being shared over an online conference call.}

        \begin{itshape}
            Thank you for taking the time today to provide some feedback on the early stages of a mobile application. The purpose of this app is to assist you with deciding where to dine out using an interactive map, filtered preferences and comparison feature.

            Today, you as a HCI expert, will complete a heuristic evaluation with a provided set of criteria to determine if the application meets a minimum standard of usability. There are two phases. The first is a basic walkthrough to get a feel for the application and the second phase is your analysis of usability based on the given heuristics.
        \end{itshape}

    \subsection*{Consent}
        \begin{itshape}
            Before we get started, please read carefully through this consent form. It reiterates the purpose for today and how your data will be used. Your personal details will not be used directly in any way and all observations are of your interaction with the software only. If you like to proceed with contributing please fill out this form and upload with the given link.
        \end{itshape}

        \textcolor{lightgray}
            {User reads through and fills out consent electronically with provided link and uploads.}
        
        \begin{itshape}
            Thanks for filling that out, please save it on your computer for the time being. If it any time you don’t wish to continue just let me know and we will stop, and none of your feedback will be used. A reminder that I am only testing the software and not evaluating you.
        \end{itshape}

\section*{HEURISTIC EVALUATION}
    \subsection*{Instructions - Phase 1}
        \begin{itshape}
            Let's get started with Phase 1. This step is to simply get a feel for the application. Select NEXT on the Google Form. On this page you will see a list of tasks to complete using the application (based on user needs) and a link to the prototype.    

            Keep this page open to refer back to. Please follow the link to open the prototype. Take your time to explore the application by performing the provided tasks which will access every feature of the application. When you feel comfortable we will move on to phase two. 
        \end{itshape}

        \textcolor{lightgray}{User is able to find the link and has no questions.}

        \begin{itshape}
            You can start. Please ask any questions you may have.
        \end{itshape}

        \textcolor{lightgray}{The user has expressed that they feel comfortable with the application at this time.}
    
    \subsection*{Instructions - Phase 2}
        \begin{itshape}    
            Now that you are more familiar with the application, it is time to look at the specific features of the application and determine their usability according to the chosen heuristics. Just as before you will complete each task while identifying issues.  For each issue you are asked to describe the issue, assign it to one or more heuristic category and rate its severity. 
            
            There are ten heuristics which you will use in your analysis. Select NEXT on the Google Form to view a link that will take you to a list of these heuristics and what they mean. On the second sheet of this link is where you will be able to fill out all relevant issues and associated information. 

            \textcolor{lightgray}{User is able to find the link and has no questions.}

            You can start. Add as many issues as you like and feel free to discuss your process.
        \end{itshape} 
    
    \subsection*{Walkthrough Tasks}
    When exploring the application, the expert will follow the steps during both phases to ensure every feature of this application is viewed and understood for analysis.
        \begin{multicols}{2}
            \setlist[enumerate]{font=\bfseries}
            \begin{itemize}        
                \item Interactive map filtered by preference
                    \begin{enumerate}
                        \item Select LOCATION, VEGETARIAN, ITALIAN, Italian, Vegetarian, \$15 - \$20, ALL MEALS.
                        \item Search the map and choose a restaurant. 
                    \end{enumerate}
                \item Filter embedded restaurant menu by preference
                    \begin{enumerate}[resume]
                        \item Scroll through the menu
                    \end{enumerate}
                \item Promote Existing Deals
                    \begin{enumerate}[resume]
                        \item Look at the deals
                    \end{enumerate}
                \item Restaurant Info
                    \begin{enumerate}[resume]
                        \item Check the opening hours of A PIZZA PLACE and go there
                    \end{enumerate}
                \item Recommend to a friend
                    \begin{enumerate}[resume]
                        \item Make a recommendation.
                        \item View all reviews of PIZZA PLACE
                    \end{enumerate}
                \item Favourite Restaurants
                    \begin{enumerate}[resume]
                        \item Favourite A PIZZA PLACE
                        \item Remove THAI LEGEND from your favourites.
                    \end{enumerate}
                \item Track User History
                    \begin{enumerate}[resume]
                        \item View A PIZZA PLACE in your history
                    \end{enumerate}
                \item Remember preferences
                    \begin{enumerate}[resume]
                        \item Set your default preferences
                    \end{enumerate}
                \item Editable and Shareable List
                    \begin{enumerate}[resume]
                        \item Add BURGER SHACK to compare.
                        \item Remove FISH \& CHIPS from compare.
                        \item Share list.
                    \end{enumerate}
            \end{itemize}
        \end{multicols}

    \subsection*{Heuristics}
        \begin{multicols}{2}
            \begin{enumerate}
                \item Provide immediate notification of application status.
                \item Use a theme and consistent terms, as well as conventions and standards familiar to user. 
                \item Prevent problems where possible; assist users should an error occur.
                \item User control and freedom.
                \item Each interface should focus on one task. 
                \item Recognition rather than recall
                \item Aesthetic and minimalist design
                \item Design a clear navigable path to task completion  
                \item Allow configuration options and shortcuts.
                \item Facilitate easier input
            \end{enumerate}
        \end{multicols}

    \subsection*{Observations}
    For each issue the following will be recorded (as outlined on the form) as a part of stage two:
        \begin{multicols}{2}
            \begin{itemize}
                \item Screen/element description
                \item Usability issue
                \item Heuristic Category
                \item Probable effect on user
                \item Severity rating
                    \begin{itemize}
                        \item Frequency - of encountering problem: rare, common
                        \item Impact - of problem: low, high
                        \item Persistence - how easy to overcome: not, very
                    \end{itemize}
            \end{itemize}  
        \end{multicols} 

    % \subsection*{Results}
    % By combining the three factors of the severity rating from the observation each issue can be rated on a four-point rating scale.
    %     \begin{center}
    %         \begin{tabular}{|c|c|c|c|c|}
    %             \hline
    %             Rating & Problem & Freq & Imp & Per \\
    %             \hline \hline
    %             0 & Not a problem at all & - & - & - \\
    %             \hline
    %             1 & Cosmetic & rare & low & not \\
    %             \hline
    %             2 & Minor & common & low & not \\
    %             \hline
    %             3 & Major & common & low & very \\
    %             \hline
    %             4 & Catastrophic & common & high & very \\
    %             \hline
    %         \end{tabular}
    %     \end{center}
 
\section*{Conclusion}
    \begin{itshape}
        All done. Thank you so much for your time today. Just a reminder that if you would like to withdraw at any time, let me know and your data will not be used. Thank you for your time, it is greatly appreciated and your data is very valuable.
    \end{itshape}

\end{document}